\subsection*{\Large Общая характеристика работы}
\fontsize{14pt}{15pt}\selectfont
\underline{\textbf{Актуальность темы.}}
В  последние десятилетия наметилось отчётливое направление развития многопроцессорных вычислительных систем в сторону
популярности массивно-параллельных систем с распределенной памятью и большим количеством узлом. Системы такого типа
превалируют среди списка TOP-500 наиболее мощных многопроцессорных вычислительных систем мира, и многопроцессорные
вычислительные комплексы, занимающие в нём верхние строки насчитывают десятки и сотни тысяч вычислительных узлов.
Развитие коммуникационных сетей делает наращивание количества узлов наиболее эффективным способм увеличения
вычислительной мощности  многопроцессорной системы. Однако, весьма небольшое количество существующих методов решения прикладных задач
обладает достаточной масштабируемостью и может быть перенесено на вычислительные системы петафлопсного диапазона без
 модификаций. Подавляюще большинство из них требует серьезной работы по адаптации реализуемых алгоритмов, и зачастую
 приводит к созданию новых, параллельных, методов.

 Задача решения СЛАУ с большими разреженными слабоструктурированными матрицами над полем $GF(2)$ играет чрезвычайно
 важную в роль в прикладных задачах криптографии, в частности, методе решета числового поля. Блочный алгоритм Ланцоша
 лежит в основе многих методов параллельного решения систем линейных алгебраических уравнений над полем $GF(2)$, однако
 его параллельные реализации  предъявляют чрезвычайно  высокие требования к коммуникационным сетям, из-за чего зачастую
 предпочтение отдается более трудоемким, но менее требовательным к коммуникационной среде методам, основанным на
 алгоритме Видеманна-Копперсмита. 

 Таким образом, оптимизация наиболее ресурсозатратного этапа алгоритма Ланцоша, а именно матрично-векторного умножения,
 является актуальной и практически значимой задачей.

\underline{\textbf{Цель работы.}} Целью диссертации является разработка математических моделей и алгоритмов для
исследования процесса итеративного умножения больших разреженных слабоструктурированных матриц на вектор на
многопроцессорных вычислительных системах с распределённой памятью для определния границ масштабируемости вышеупомянутых
методов и поиска оптимальных параметров распределения данных по узлам вычислительной системы для повышения
масштабируемости.

\underline{\textbf{Научная новизна, теоретическая и практическая значимость.}}
\begin{enumerate}
 \item Предложена математическая модель процесса параллельного итеративного матрично-векторного умножения в
   многопроцессорных вычислительных системах с распределенной памятью, позволяющая эффективно учесть особенности данного
   процесса для больших слабоструктурированных разреженных матриц. Предлагаемая модель позволяет найти приближённые
   значения оптимальных параметров запуска итеративного процесса матричного-векторного умножения с учетом  характристик
   задачи и используемой вычислительной системы.
 \item Разработан параллельный код для моделирования процесса итеративного матрично-векторного умножения, поддерживающий
   генерацию синтетических тестовых данных с широким спектром характеристик, и позволяющий моделировать различные
   варианты использования ресурсов многопроцессорной вычислительной системы и различные варианты распределния данных по
   узлам многопроцессорной вычислительной системы.
 \item Разработан программный комплекс, позволяющий серий численных экспериментов уточнить оценки параметров, получаемые
   при помощи предлагаемой математической модели.  Проведена практическая демонстрация эффективности разработанных
   программных средств при исследовании способов распределения данных в коммуникационной среде в процессе
   матрично-векторного умножения больших слабоструктурировнаных матриц. 
\end{enumerate}

\underline{\textbf{Апробация работы.}}

\underline{\textbf{Публикации.}}

\underline{\textbf{Объем и структура работы.}} Диссертация состоит из~введения, трех глав и заключения. Полный объем диссертации \textbf{ХХХ}~страниц текста с~\textbf{ХХ}~рисунками и~\textbf{ХХ}~таблицами. Список литературы содержит \textbf{ХХX}~наименование.

%\newpage
\subsection*{\Large Содержание работы}
Во \underline{\textbf{введении}} раскрывается актуальность темы работы, приводится краткое описание проблем и
результатов, относящихся к теме диссертации. Приведен обзор существующих моделей и средств повышения эффективности
параллельного итеративного матрично-векторного умножения. Кратко излагается структура и содержание работы по главам и
основные полученные результаты. 

\underline{\textbf{Первая глава}} 

\underline{\textbf{Вторая глава}}  

\underline{\textbf{Третья глава}}  

В \underline{\textbf{заключении}} приведены основные результаты работы, которые заключаются в следующем:
\begin{enumerate}
 \item Предложена математическая модель процесса параллельного итеративного матрично-векторного умножения в
   многопроцессорных вычислительных системах с распределенной памятью, позволяющая эффективно учесть особенности данного
   процесса для больших слабоструктурированных разреженных матриц.  
 \item  Создан комплекс программных средств для численного моделирования процесса итеративного матрично-векторного умножения, поддерживающий  генерацию синтетических тестовых данных с широким спектром характеристик, и позволяющий моделировать различные
   варианты использования ресурсов многопроцессорной вычислительной системы и различные варианты распределния данных по
   узлам многопроцессорной вычислительной системы.

 \item   Проведено исследовании способов определения оптимальных параметров распределения данных в коммуникационной среде в процессе
   матрично-векторного умножения больших слабоструктурировнаных матриц, предложенны методы, позволяющие повысить
   масштабируемость данного процесса.
 
\end{enumerate}


%\newpage
\renewcommand{\refname}{\Large Публикации автора по теме диссертации}
\nocite{*}
\bibliography{biblio}
